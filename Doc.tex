\documentclass[14pt, a4paper]{extarticle}
\usepackage{booktabs}
\usepackage{adjustbox}
%\usepackage{graphicx}
%\graphicspath{{/home/Prometheus/assignments/SSAD/OWP}}

\title{\emph{\textbf{Databases Documentation}:}}
\author{Parth Laxmikant Kolekar, 201301143\\
Aaditya M Nair, 201302161 }

\begin{document}
    \maketitle

    \section{About the Project}
        The project is about a pharmacy system. The system stores data about 
        medicines, every sale, and also maintains an inventory of medicines 
        available. The project is divided into 3-tiers:
        \begin{enumerate}
                \item The Front-End \\
                \item Request Handling Scripts \\
                \item SQL Database \\
        \end{enumerate}

    \section{SQL Database}
        Database sever used for the backend is a MySQL Server. A database was created 
        with the following tables:
        \begin{enumerate}
                \item Suppliers \\
                \item Medicines \\
                \item Inventory \\
                \item Patient \\
                \item Patient Contact \\
                \item Sales \\
                \item Medicines Sold \\
                \item Employees \\
                \item Employee Contact \\
                \item etc\ldots \\
        \end{enumerate}

        In addition to this we also created a "user" database to store usernames and
        passwords of all users. The users are linked with the Employee Database.

        The passwords are stored using Industry Standard encryption algorithms.
        The password is hashed using the \emph{SHA256} hashing algorithm (one of 
        the most secure hashing algorithm) and salted using a randomly generated 
        string encoded into 64-bits. The hashed and salted password along with 
        the username and salt are stored in the users database.

    \section{Request Handling Scripts}
        All of this section of code is written in PHP\@. Functions are used to
        interact with the MySQL database. All SQL are prepared and sanitised 
        before their sending and execution. Hence SQL- Injection attacks are 
        vastly mitigated this way. Different functions are used for INSERT and
        UPDATE so that User level tasks can easily be seperated from admin level
        tasks. This allows us for seperation of concerns.

    \section{Front-End}
        Front-End is developed using simple HTML5, JavaScript and occasionally, 
        PHP\@. The front-end simply displays the database in tabular format. CRUD
        operations are possible on that table which directly get updated.
        There are seperate conrols for general users and admins. Access is granted only
        to authorised users via the login system. 
        
\end{document}

















